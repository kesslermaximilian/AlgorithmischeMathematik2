\lecture{1}{Mo 12 Apr 2021 10:16}{Grundbegriffe}

\begin{itemize}
    \item Es gibt ein Helpdesk, auch explizit für Studentinnen
    \item die Vorlesung wird aufgenommen, und zwar ohne Videos der Teilnehmenden sowie des Dozenten, die Aufzeichnung werden anschließend in Sciebo hochgeladen.
    \item Es gibt ein Diskussionsforum für Fragen (auf eCampus).
    \item Ab heute Abend, 18 Uhr (Mo 12 Apr 2021 18:00), kann man sich auf eCampus für die Übungsgruppen registrieren und endet am Dienstag Abend um 24 Uhr (Di 12 Apr 2021 24:00), es wird versucht, die Studenten gleichmäßig zu verteilen.
    \item Falls ihr in der Warteliste landet und gewünscht ist, in der Gruppe abzugeben, schreibt eine Mail mit den gewünschten Abgabepartner, dann kann eine gemeinse Einteilung erfolgen.
    \item Es gibt auch das Modul \verb?AlmaIIb?. Registriert euch noch nicht, dies ist für den 2. Teil der Vorlesung notwendig. 
    \item Die Abgabe der Übungsblätter erfolgt einheitlich jeden Freitag um 12 Uhr.
    \item Gruppenabgaben sind erlaubt, bis zu einer Größe von maximal 4 StudentInnen.
    \item Das 1. Blatt ist freiwillig und gibt Bonuspunkte.
    \item Für die Klausurzulassung werden 50\% der Punkte benötigt. Von den Programmieraufgaben müssen mindestens 4 von 6 zufriedenstellend bearbeitet werden.
    \item Programmieraufgaben gibt es ab dem 2. Übungsblatt auf jedem 2. Blatt. Die Bearbeitungszeit beträgt dann 2 Wochen.
\end{itemize}


\section*{Einleitung}
In der Vorlesung werden wir sehen:
\begin{description}
    \item[Teil 1: Diskrete Stochastik]%
        \begin{itemize}
            \item Zufallsvariablen
            \item Bedingte Wahrscheinlichkeiten
            \item Unabhängigkeit von Variablen
            \item Monte-Carlo Methoden
        \end{itemize}
    \item[Teil 2: Numerische Analysis]
        \begin{itemize}
            \item Iterative Verfahren
            \item Interpolation von Daten (durch Polynome, trigonometrische Funktionen, \ldots)
            \item Numerische Verfahren für die Integration
        \end{itemize}
\end{description}


\section{Diskrete Stochastik}
\subsection{Einleitung}
\begin{goal}
    Beschreibung von Systemen, die einen Anteil an \vocab{Zufall} haben, d.h. nicht 100\% deterministisch sind.
\end{goal}
\begin{example}
    \begin{itemize}
        \item Spiele: Kartenspiele, Glücksspiele, \ldots
        \item Statistik: Umfragen, Versicherung
        \item Komplexe Systeme: Wettermodelle, Finanzmärkte
    \end{itemize}
\end{example}

\underline{Was sind Quellen von Zufall?}
\begin{itemize}
    \item Zu komplexe Systeme. Dann sieht der Gesamteffekt zufällig aus.
    \item Fehlende Informationen (z.B. bei einem Kartenspiel)
    \item Chaotische Systeme (Wetter
    \item Intrinsisch unvorhersagbare Systeme (z.B. radioaktiver Zerfall)
\end{itemize}
\begin{question}
    \begin{enumerate}[(1)]
        \item Wie modelliert man ein System mit Zufall?
        \item Wie simuliert man ein System mit Zufall? (anwendungstechnischer)
        \item Welche Voraussagen kann man machen?
    \end{enumerate}
\end{question}


\begin{example}
    Die \vocab{Brown'sche Bewegung}. Das System ist implizit ein Pollen mit vielen Wassermolekülen ($\sim 10^{23})$, die sich im Prinzip deterministisch bewegen. \\
    $\implies$ Wir erhalten ein Gleichungssystem mit $(N+1)\cdot 6$ (3 Positionen, 3 Geschwindigkeit) Variablen. Dieses ist de facto unlösbar. \\

    Was wollen wir hier eigentlich untersuchen? -> Die Bewegung des Pollens, jedoch nicht die der einzelnen Wassermoleküle. \\
    In einer \vocab{Modellierung} ersetzt man die Stöße, die ,durch die Wassermoleküle entstehen durch \vocab{zufällige Stöße}. 
\end{example}

\underline{Diskretes Modell:} Die Zeit bewegt sich in $n\in \left \{0,1,2,\ldots\right\} $. Sei
\[
    Z(n) := (\text{Position des Pollens zur Zeit $n$}) \in  \Z^3
.\] 
OBdA setzen wir $Z(0) = 0$. \\
\underline{Dynamik}: $Z(n+1) = Z(n) + \xi_n$, wobei wir $\xi_n$ aus dem Ergebnis eines Würfelwurfs bestimmen werden:
 \[
\xi_n = \begin{cases}
    (1,0,0) & \text{wenn Würfel}=1 \\
    (-1,0,0) & \text{wenn Würfel}=2 \\
    (0,1,0) & \text{wenn Würfel}=3 \\
    (0,-1,0) & \text{wenn Würfel}=4 \\
    (0,0,1) & \text{wenn Würfel}=5 \\
    (0,0,-1) & \text{wenn Würfel}=6 \\

\end{cases}
.\] 

\begin{question}
    Welche Fragen können wir mit solch einem System nun beantworten? Was pasiert, wenn $n\gg 1$.
\end{question}
\begin{enumerate}[\protect\circled{\alph*}]
    \item Typischerweise erhalten wir $\abs{Z(n)} =  O(\sqrt{n}) $ 
    \item Wenn wir die Frequenz von $[Z(n)]_i$ betrachten, sehen wir typischerweise: \\
        (Füge Graph mit grober Binomialverteilung ein)\\
        Für $n\gg 1$ sieht diese Verteilung dann ungefähr wie die Gaussglocke aus. \\
\end{enumerate}
\underline{Skalierung:} Wir setzen nun
\[
    B(t) = \lim_{n \to \infty} \frac{Z(\left\lfloor nt \right\rfloor )}{\sqrt{n} }
.\] 
und dies ist dann die Brownsche Bewegung.

\begin{question}
    \begin{itemize}
        \item Ist $Z(n)$ in einer gegebenen Menge  $A$? \\
    -> Im Allgemeinen kann man das nicht mit 'Ja' oder 'Nein' beantworten.
\item Wenn man $Z(n)$ beobachtet, wie häufig wird  $Z(n)$ in  $A$ sein?\\
    -> Diese Frage lässt sich mit einer Zahl $\in [0,1]$ beantworten.
    \end{itemize}
\end{question}

\subsection{Ereignisse und Wahrscheinlichkeiten}
Wir benötigen 3 Grundelemente:
\begin{enumerate}[(1)]
    \item Die Menge $\Omega$ von möglichen \vocab{Ergebnissen}. die Elemente von $\Omega$ heißen auch  \vocab{Elementarereignisse}.
    \item Die Menge $\mathcal{F}$ der \vocab{Ereignisse}. Ein Ereignis  $E$ ist eine Eigenschaft, die an einer Teilmenge von $G\subset \Omega$ assoziiert ist: $ω\in G \iff $ Eigenschaft $E$ ist erfüllt.
    \item Eine \vocab{Wahrscheinlichkeitsverteilung (auch W-maß)}:
        \[
            \mathbb{P}: \mathcal{F} \to  [0,1]
        .\] 
\end{enumerate}
\begin{remark}
    Wir werden noch sehen, dass gewisse Dinge für unsere Begriffe erfüllt sein müssen, dazu aber später mehr.
\end{remark}
\begin{example}
    Eine Urne hat 12 nummerierte Kugeln (von 1 bis 12).
    \begin{enumerate}[(1)]
        \item Das \underline{Zufallsexperiment} besteht daraus, dass wir eine Kugel aus der Urne ziehen und die Zahl notieren, die wir sehen. D.h.
            \[
            \Omega = \left \{1,\ldots,12\right\} 
            .\] 
            Ein Elementarereignis ist nun z.B. gegeben durch $ω = \left \{5\right\}  \equiv 5$ (wir vereinfachen die Notation).
        \item Mögliche Ereignisse sind z.B:
            \begin{equation}
                \begin{split}
                    A &= \text{"Die Zahl ist gerade} \\
                    B&= \text{"Die Zahl ist }\leq 5 \\
                    C &= \text{"Die Zahl ist 8}
                \end{split}
            \end{equation}
            Die assoziierten Mengen sind dann
            \begin{equation}
                \begin{split}
                    A &= \left \{2,4,6,8,10,12\right\}  \\
                     B &= \left \{1,2,3,4,5\right\} \\
                      C & = \left \{8\right\} 
                \end{split}
            \end{equation}
        \item Für die Wahrscheinlichkeiten nehmen wir an, dass jede Kugel die gleiche Chance hat, gezogen zu werden, d.h.
            \[
                \forall G\in \mathcal{F}: \mathbb{P}(G) = \frac{\abs{G}}{\abs{\Omega} }
            .\] 
            Wir erhalten als
            \[
                \mathbb{P}(A) = \frac{6}{12}=\frac{1}{2} \qquad \mathbb{P}(B) = \frac{5}{12} \qquad \mathbb{P}(C) = \frac{1}{12}
            .\] 
    \end{enumerate}
\end{example}

\begin{notation}
    $A\equiv \left \{ω\in \Omega \mid  ω\in A\right\} \equiv  \left \{ω\in A\right\} \equiv \left \{A \text{ tritt ein}\right\} $
\end{notation}






