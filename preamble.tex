%Preamble Mainly taken from
%https://github.com/gillescastel/university-setup/blob/master/preamble.tex
%but edited for own purposes

%%%%%OWN
%%Own basic packages
\usepackage[subfigure]{tocloft}
\newcommand\listlecturename{Übersicht der Vorlesungen}

\newlistof{lecture}{lec}{\listlecturename}

\renewcommand{\cftdot}{}
\renewcommand\cftlecturefont{\normalfont\bfseries}
\renewcommand\cftlecturepagefont{\normalfont\bfseries}
\renewcommand\cftlectitlefont{\Large\bfseries}

\renewcommand\cftbeforelectureskip{1em}

\usepackage{mkessler-math}
\usepackage[lecturenumbers]{mkessler-fancythm} %%Options available:
%lecturenumbers -> number theorems like in the lecture (default, even it ommited)
%truenumbers -> number all theorems (corollarys, lemmata, etc), even if not numbered in the lecture
%numberall -> number examples, notation, remark, ... as well (and all theorems etc).
%Choose as you like
\usepackage{mkessler-operators}

\usepackage{datetime}
\date{{\normalfont Mitschrift}\\{\sc Maximilian Keßler}}

\addbibresource{bibliography.bib}

%%%%%%Preamble from Gilles Castel
% Some basic packages
\usepackage{url}
\usepackage{graphicx}
\usepackage{float}

% for wrapping text around figures
\usepackage{wrapfig}

%%This option is for now commented out, not sure what it does, but causes errors
%\pdfminorversion=7


% Don't indent paragraphs, leave some space between them
\usepackage{parskip}

% Hide page number when page is empty
\usepackage{emptypage}
% Other font I sometimes use.
\usepackage{xcolor}
% \usepackage{cmbright}

% Math stuff
\usepackage{amsfonts}

% Put x \to \infty below \lim
\let\svlim\lim\def\lim{\svlim\limits}

%Make implies and impliedby shorter
\let\implies\Rightarrow
\let\impliedby\Leftarrow
\let\iff\Leftrightarrow
\let\epsilon\varepsilon

% Command for short corrections
% Usage: 1+1=\correct{3}{2}

% Environments
\makeatother

% Fix some spacing
% http://tex.stackexchange.com/questions/22119/how-can-i-change-the-spacing-before-theorems-with-amsthm
\makeatletter
\def\thm@space@setup{%
  \thm@preskip=\parskip \thm@postskip=0pt
}


% \lecture starts a new lecture (les in dutch)
%
% Usage:
% \lecture{1}{di 12 feb 2019 16:00}{Inleiding}
%
% This adds a section heading with the number / title of the lecture and a
% margin paragraph with the date.

% I use \dateparts here to hide the year (2019). This way, I can easily parse
% the date of each lecture unambiguously while still having a human-friendly
% short format printed to the pdf.

\usepackage{xifthen}
\def\testdateparts#1{\dateparts#1\relax}
\def\dateparts#1 #2 #3 #4 #5\relax{
    \marginpar{\small\textsf{\mbox{#1 #2 #3 #5}}}
}

\def\@lecture{}%
\newcommand{\lecture}[3][]{
    \refstepcounter{lecture}
    \ifthenelse{\isempty{#3}}{%
    \def\@lecture{\ifenglish Lecture\else Vorlesung\fi\, \thelecture}%
    }{%
    \def\@lecture{\ifenglish Lecture\else Vorlesung\fi\, \thelecture: #3}%
    }%
    \marginpar{\small\textsf{\parbox{10em}{\ifenglish Lecture \else Vorlesung\fi\, \thelecture \\#2}}}
    \addcontentsline{lec}{lecture}{\ifenglish Lecture \else Vorlesung\fi \,\thelecture\,(#2)}
    \ifthenelse{\isempty{#1}}{}{\addtocontents{lec}{\smallskip\hspace{1.5em}\protect\parbox{\dimexpr\textwidth-\@pnumwidth - 7em}{#1}}}
}

% These are the fancy headers
\usepackage{fancyhdr}
\pagestyle{fancy}

% LE: left even
% RO: right odd
% CE, CO: center even, center odd
% My name for when I print my lecture notes to use for an open book exam.
% \fancyhead[LE,RO]{Gilles Castel}

\fancyhead[RO,LE]{\@lecture} % Right odd,  Left even
\fancyhead[RE,LO]{}          % Right even, Left odd

\fancyfoot[RO,LE]{\thepage}  % Right odd,  Left even
\fancyfoot[RE,LO]{}          % Right even, Left odd
\fancyfoot[C]{\leftmark}     % Center

\makeatother

% Todonotes and inline notes in fancy boxes
\usepackage{todonotes}

% Make boxes breakable
\tcbuselibrary{breakable}

% Figure support as explained in my blog post.
\usepackage{import}
\usepackage{xifthen}
\usepackage{pdfpages}
\usepackage{transparent}
\newcommand{\incfig}[1]{%
    \def\svgwidth{\columnwidth}
    \import{./figures/}{#1.pdf_tex}
}

% Fix some stuff
% %http://tex.stackexchange.com/questions/76273/multiple-pdfs-with-page-group-included-in-a-single-page-warning
\pdfsuppresswarningpagegroup=1





