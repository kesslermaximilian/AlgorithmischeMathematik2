%Partially taken from
%https://github.com/gillescastel/university-setup/blob/master/preamble.tex
%but edited for own purposes

% Some basic packages
\usepackage[ngerman]{babel}
\usepackage[T1]{fontenc}
\usepackage[mathletters]{ucs}
\usepackage[utf8x]{inputenc}
\usepackage{textcomp}
\usepackage{url}
\usepackage{graphicx}
\usepackage{float}
\usepackage{booktabs}
\usepackage[shortlabels]{enumitem}
\usepackage{hyperref}

\makeatletter
\newcommand\setItemnumber[1]{\setcounter{enum\romannumeral\@enumdepth}{\numexpr#1-1\relax}}
\makeatother

% for wrapping text around figures
\usepackage{wrapfig}


\pdfminorversion=7

% Don't indent paragraphs, leave some space between them
\usepackage{parskip}

% Hide page number when page is empty
\usepackage{emptypage}
\usepackage{subcaption}
\usepackage{multicol}
% Other font I sometimes use.
\usepackage{xcolor}
\usepackage{comment}
% \usepackage{cmbright}

% Math stuff
\usepackage{amsmath, amsfonts, mathtools, amsthm, amssymb}
% Fancy script capitals
\usepackage{mathrsfs}
\usepackage{cancel}
% Bold math
\usepackage{bm}
% Some shortcuts
\newcommand\N{\ensuremath{\mathbb{N}}}
\newcommand\R{\ensuremath{\mathbb{R}}}
\newcommand\Z{\ensuremath{\mathbb{Z}}}
\renewcommand\O{\ensuremath{\emptyset}}
\newcommand\Q{\ensuremath{\mathbb{Q}}}
\newcommand\C{\ensuremath{\mathbb{C}}}

% Easily typeset systems of equations (French package)
\usepackage{systeme}

% Put x \to \infty below \lim
\let\svlim\lim\def\lim{\svlim\limits}

%Make implies and impliedby shorter
\let\implies\Rightarrow
\let\impliedby\Leftarrow
\let\iff\Leftrightarrow
\let\epsilon\varepsilon

% Add \contra symbol to denote contradiction
\usepackage{stmaryrd} % for \lightning
\newcommand\contra{\scalebox{1.5}{$\lightning$}}

% \let\phi\varphi

% Command for short corrections
% Usage: 1+1=\correct{3}{2}

\definecolor{correct}{HTML}{009900}
\newcommand\correct[2]{\ensuremath{\:}{\color{red}{#1}}\ensuremath{\to }{\color{correct}{#2}}\ensuremath{\:}}
\newcommand\green[1]{{\color{correct}{#1}}}

% horizontal rule
\newcommand\hr{
    \noindent\rule[0.5ex]{\linewidth}{0.5pt}
}

% hide parts
\newcommand\hide[1]{}

% si unitx
\usepackage{siunitx}
\sisetup{locale = FR}

%Theorem-environments
\usepackage{mdframed}
\mdfsetup{skipabove=\topskip,skipbelow=\topskip}

\newtheoremstyle{own}%〈name〉
{3pt}%〈Space above〉1
{3pt}%〈Space below〉1
{}%〈Body font〉
{}%〈Indent amount〉2
{\bfseries}%〈Theorem head font〉
{.}%〈Punctuation after theorem head〉
{.5em}%〈Space after theorem head〉3
{}%〈Theorem head spec(can be left empty, meaning ‘normal’)〉

\theoremstyle{own}

\global\mdfdefinestyle{thm}{linecolor=red,linewidth=2pt,leftmargin=0cm,rightmargin=0cm, backgroundcolor=red!8, rightline=false, topline=false, bottomline=false}

\global\mdfdefinestyle{lemma}{linecolor=orange,linewidth=2pt,leftmargin=0cm,rightmargin=0cm, backgroundcolor=orange!10, rightline=false, topline=false, bottomline=false}

\global\mdfdefinestyle{definition}{linecolor=blue,linewidth=2pt,leftmargin=0cm,rightmargin=0cm, backgroundcolor=blue!7, rightline=false, topline=false, bottomline=false}

\global\mdfdefinestyle{example}{linecolor=green!70!black,linewidth=2pt,leftmargin=0cm,rightmargin=0cm, rightline=false, topline=false, bottomline=false}

\global\mdfdefinestyle{remark}{linecolor=yellow!80!orange,linewidth=2pt,leftmargin=0cm,rightmargin=0cm, rightline=false, topline=false, bottomline=false}


\global\mdfdefinestyle{notation}{linecolor=violet,linewidth=2pt,leftmargin=0cm,rightmargin=0cm, backgroundcolor=violet!7, rightline=false, topline=false, bottomline=false}

\global\mdfdefinestyle{theoremdef}{linecolor=red,linewidth=2pt,leftmargin=0cm,rightmargin=0cm, backgroundcolor=blue!7, rightline=false, topline=false, bottomline=false}

\newtheorem{protothm}{Theorem}[section]
\newenvironment{theorem}{\begin{mdframed}[style=thm] \begin{protothm}}{\end{protothm}\end{mdframed}}

\newtheorem*{protothm*}{Theorem}
\newenvironment{theorem*}{\begin{mdframed}[style=thm] \begin{protothm*}}{\end{protothm*}\end{mdframed}}

\newtheorem{protoprop}[protothm]{Proposition}
\newenvironment{proposition}{\begin{mdframed}[style=thm] \begin{protoprop}}{\end{protoprop}\end{mdframed}}



\newtheorem{protocor}[protothm]{Corollary}
\newenvironment{corollary}{\begin{mdframed}[style=thm] \begin{protocor}}{\end{protocor}\end{mdframed}}

\newtheorem{protolemma}[protothm]{Lemma}
\newenvironment{lemma}{\begin{mdframed}[style=lemma] \begin{protolemma}}{\end{protolemma}\end{mdframed}}

\newtheorem*{protlemma*}{Lemma}
\newenvironment{lemma*}{\begin{mdframed}[style=lemma] \begin{protolemma*}}{\end{protolemma*}\end{mdframed}}

\newtheorem{protodefinition}[protothm]{Definition}
\newenvironment{definition}{\begin{mdframed}[style=definition] \begin{protodefinition}}{\end{protodefinition}\end{mdframed}}


\newtheorem{prototheoremdef}[protothm]{Theorem and Definition}
\newenvironment{theoremdef}{\begin{mdframed}[style=theoremdef] \begin{prototheoremdef}}{\end{prototheoremdef}\end{mdframed}}


\newtheorem*{protodefinition*}{Definition}
\newenvironment{definition*}{\begin{mdframed}[style=definition] \begin{protodefinition*}}{\end{protodefinition*}\end{mdframed}}

\newtheorem*{protonotation}{Notation}
\newenvironment{notation}{\begin{mdframed}[style=notation] \begin{protonotation}}{\end{protonotation}\end{mdframed}}

\newtheorem*{protoexample}{Example}
\newenvironment{example}{\begin{mdframed}[style=example] \begin{protoexample}}{\end{protoexample}\end{mdframed}}

\newtheorem*{protoremark}{Remark}
\newenvironment{remark}{\begin{mdframed}[style=remark] \begin{protoremark}}{\end{protoremark}\end{mdframed}}

% Recaps
\usepackage[skins]{tcolorbox}

\newtcolorbox{recap}{before skip = 0.5cm, after skip = 0.5cm, enhanced, sharp corners = all, colback = white, colframe = gray, toprule=0pt, bottomrule=0pt, leftrule=0pt,rightrule=0pt, overlay = {
        \draw[gray, line width = 2pt] (frame.north west) -- ++(0.5cm, 0pt);
        \draw[gray, line width=2pt] (frame.south east) -- ++(-0.5cm, 0pt);
        \draw[gray, line width=2pt] (frame.north west) -- ++ (0pt, -0.5cm);
        \draw[gray, line width=2pt] (frame.south east) -- ++(0pt, 0.5cm);
}}



% Environments
\makeatother
% For box around Definition, Theorem, \ldots
\theoremstyle{definition}
\newtheorem*{problem}{Problem}

\newtheorem*{previouslyseen}{As previously seen}
\newtheorem*{warning}{\color{red} Warning}
\newtheorem*{info}{Information}
\newtheorem*{claim}{Claim}
\newtheorem*{moral}{Moral}
\newtheorem*{tday}{Plan for today}
\newtheorem*{aim}{Ziel}
\newtheorem*{summary}{So far}
\newtheorem*{fact}{Fact}
\newtheorem*{exercise}{Exercise}
\newtheorem*{nxt}{Next}
\newtheorem*{note}{Note}
\newtheorem*{question}{Question}
\newtheorem*{answer}{Answer}
\newtheorem*{observe}{Observe}
\newtheorem*{property}{Property}
\newtheorem*{intuition}{Intuition}
\newtheorem*{goal}{Goal}
\newtheorem*{recall}{Recall}
\newtheorem*{idea}{Idea}

% End example and intermezzo environments with a small diamond (just like proof
% environments end with a small square)
\usepackage{etoolbox}
\AtEndEnvironment{example}{\null\hfill$\diamond$}%
% \AtEndEnvironment{opmerking}{\null\hfill$\diamond$}%

% Fix some spacing
% http://tex.stackexchange.com/questions/22119/how-can-i-change-the-spacing-before-theorems-with-amsthm
\makeatletter
\def\thm@space@setup{%
  \thm@preskip=\parskip \thm@postskip=0pt
}


% Exercise 
% Usage:
% \oefening{5}
% \suboefening{1}
% \suboefening{2}
% \suboefening{3}
% gives
% Oefening 5
%   Oefening 5.1
%   Oefening 5.2
%   Oefening 5.3
\newcommand{\oefening}[1]{%
    \def\@oefening{#1}%
    \subsection*{Oefening #1}
}

\newcommand{\suboefening}[1]{%
    \subsubsection*{Oefening \@oefening.#1}
}


% \lecture starts a new lecture (les in dutch)
%
% Usage:
% \lecture{1}{di 12 feb 2019 16:00}{Inleiding}
%
% This adds a section heading with the number / title of the lecture and a
% margin paragraph with the date.

% I use \dateparts here to hide the year (2019). This way, I can easily parse
% the date of each lecture unambiguously while still having a human-friendly
% short format printed to the pdf.

\usepackage{xifthen}
\def\testdateparts#1{\dateparts#1\relax}
\def\dateparts#1 #2 #3 #4 #5\relax{
    \marginpar{\small\textsf{\mbox{#1 #2 #3 #5}}}
}

\def\@lecture{}%
\newcommand{\lecture}[3]{
    \ifthenelse{\isempty{#3}}{%
        \def\@lecture{Lecture #1}%
    }{%
        \def\@lecture{Lecture #1: #3}%
    }%
    \subsection*{\@lecture}
    \marginpar{\small\textsf{\mbox{#2}}}
}



% These are the fancy headers
\usepackage{fancyhdr}
\pagestyle{fancy}

% LE: left even
% RO: right odd
% CE, CO: center even, center odd
% My name for when I print my lecture notes to use for an open book exam.
% \fancyhead[LE,RO]{Gilles Castel}

\fancyhead[RO,LE]{\@lecture} % Right odd,  Left even
\fancyhead[RE,LO]{}          % Right even, Left odd

\fancyfoot[RO,LE]{\thepage}  % Right odd,  Left even
\fancyfoot[RE,LO]{}          % Right even, Left odd
\fancyfoot[C]{\leftmark}     % Center

\makeatother




% Todonotes and inline notes in fancy boxes
\usepackage{todonotes}
\usepackage{tcolorbox}

% Make boxes breakable
\tcbuselibrary{breakable}

% Verbetering is correction in Dutch
% Usage: 
% \begin{verbetering}
%     Lorem ipsum dolor sit amet, consetetur sadipscing elitr, sed diam nonumy eirmod
%     tempor invidunt ut labore et dolore magna aliquyam erat, sed diam voluptua. At
%     vero eos et accusam et justo duo dolores et ea rebum. Stet clita kasd gubergren,
%     no sea takimata sanctus est Lorem ipsum dolor sit amet.
% \end{verbetering}
\newenvironment{verbetering}{\begin{tcolorbox}[
    arc=0mm,
    colback=white,
    colframe=green!60!black,
    title=Opmerking,
    fonttitle=\sffamily,
    breakable
]}{\end{tcolorbox}}

% Noot is note in Dutch. Same as 'verbetering' but color of box is different
\newenvironment{noot}[1]{\begin{tcolorbox}[
    arc=0mm,
    colback=white,
    colframe=white!60!black,
    title=#1,
    fonttitle=\sffamily,
    breakable
]}{\end{tcolorbox}}




% Figure support as explained in my blog post.
\usepackage{import}
\usepackage{xifthen}
\usepackage{pdfpages}
\usepackage{transparent}
\newcommand{\incfig}[1]{%
    \def\svgwidth{\columnwidth}
    \import{./figures/}{#1.pdf_tex}
}

% Fix some stuff
% %http://tex.stackexchange.com/questions/76273/multiple-pdfs-with-page-group-included-in-a-single-page-warning
\pdfsuppresswarningpagegroup=1


% My name
\author{Maximilian Keßler}




\DeclarePairedDelimiter\abs{\lvert}{\rvert}
\DeclareMathOperator{\id}{id}
\DeclareMathOperator{\coker}{coker}
\DeclareMathOperator{\dz}{dz}
\DeclareMathOperator{\diam}{diam}
\DeclareMathOperator{\ex}{ex}
\DeclareMathOperator{\dt}{dt}
\DeclareMathOperator{\dist}{dist}
\DeclareMathOperator{\Ext}{Ext}
\DeclareMathOperator{\Tor}{Tor}
\DeclareMathOperator{\supp}{supp}
\DeclareMathOperator{\im}{im}
\DeclareMathOperator{\Lip}{Lip}
\DeclareMathOperator{\Mspec}{MaxSpec}
\DeclareMathOperator{\Proj}{Proj}
\DeclareMathOperator{\QCoh}{QCoh}
\renewcommand\Im\im
\DeclareMathOperator{\Mor}{Mor}
\DeclareMathOperator{\Hom}{Hom}
\DeclareMathOperator{\Gal}{Gal}
\DeclareMathOperator{\MaxSpec}{MaxSpec}
\DeclareMathOperator{\Aut}{Aut}
\DeclareMathOperator{\dy}{dy}
\DeclareMathOperator{\Fun}{Fun}
\DeclareMathOperator{\Presh}{Pre-Sh}
\DeclareMathOperator{\Sh}{Sh}
\DeclareMathOperator{\dif}{diff}
\DeclareMathOperator{\opp}{opp}
\DeclareMathOperator{\Ob}{Ob}

\newcommand*\circled[1]{\tikz[baseline=(char.base)]{
            \node[shape=circle,draw,inner sep=2pt] (char) {#1};}}

            \usepackage[ngerman,ruled,vlined]{algorithm2e}

\newcommand{\vocab}[1]{\textbf{\color{blue} #1}}

\usepackage{tikz-cd}
\newcommand{\cat}[1]{ \mathscr{#1} }

